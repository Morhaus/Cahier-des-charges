\pagenumbering{arabic}
\tableofcontents

\chapter{Introduction}

\paragraph{}\FR~is a mix between a car game and a soccer game. Inspired by the likes of \emph{Rocket League} and \emph{Burnout: Paradise}, we want to create our own variation of the genre. When it comes to car games, as soon as you leave the beaten path of the racing genre, there’s still a lot of uncharted territory. With our debut on the video game scene, \FR, we intend on filling that gap.\\


A game of \FR~is set in an arena where two or more teams are facing off. In the main game mode, not unlike soccer, the teams have to fight for the possession of a ball, while at the same time defending their own goal. After a given duration, whichever team scored the most goal in the enemy goal wins the game. While simplistic, this set of rules allows for the most freedom mechanics-wise. Other game modes might include playing a variant of volleyball or a vehicle deathmatch.\\


We intend to take full advantage of the powerful 3D features of Unity in order to kickstart a simple but functional game, with a focus set on mechanics and playability. While the underlying concept is simple, the execution is what matters most. The game will heavily rely on realistic physics simulation, occurring in real time across a multiplicity of clients. How well we've come to balance and tweak our set of features will directly influence the feel of the game, much more so than simply how much content we've managed to fit in it.\\


Hitherto, none of us has ever done a project the scale of \FR. It is pretty new to us, although some of us have worked on big projects in the past. So it will obviously bring us a lot of knowledge and teach us new ways of programming. Mixing a car game and a soccer game involves a lot of physics: the ball and its bounciness, the cars and their collisions with walls or even other cars, a camera that should follow the car without moving through walls.\\


Those aspects are not to take lightly. A project this big will also strengthen our friendship: working together for a whole semester can either be a bonding experience, or a death trap for friendship. Having already spent a semester together, we are confident in the robustness of our group.\\


But that is not all: deadlines, presentations, book of specifications... Three of us have faced such odds before. One might think that it becomes easier with time, when it really doesn’t. A game is a huge undertaking. There is always room for improvement, features, gameplay tweaks, designs, storylines, etc. But oftentimes, when you begin to add something new, it opens the road to new possibilities. It is up to you to decide when to stop and when to begin something new, whilst keeping the deadline in mind.\\

In short, this project is sure to bring us a lot of knowledge, discipline, strictness and cohesion.

\chapter{Team Explanation}

\paragraph{}The French Lions was created the 8th of November 2015. A group of 4 friends, who met at EPITA on the very first day of school. The name of the group emerged as evident to us, after a long-running private joke. At first, there was Simon’s hair, that looked much like a lion’s mane. Obviously, we proceeded to call him The Lion. A few weeks later, Alexandre, in a heated debate, confidently asserted that lions were indigenous to France a long time ago. A couple of hours of googling and fact-checking later, Alexandre admitted that his belief was nonsensical. This long argument stood in memories, and ultimately lead to the name we know and love, The French Lions.\\


\begin{itemize}
    \item \textbf{Alexandre Kirszenberg:}{
    Alexandre is the oldest lion. He is the alpha male of the tribe. With the absence of females in the tribe, he has to settle for Simon. After discovering programming at the age of 15, he’s made it his duty to write the best software ever conceived. When he’s not delusional, he likes to play video games, read books and watch movies.}\\


    \item \textbf{Simon Goetz:}{
    Simon, is a 16 years old boy coming from Strasbourg. He skipped two grades. He is the youngest lion and has a huge 4 years difference with the other members. He is creative and is the one that has wasted the longer time playing video games, so he can bring to The French Lions ideas for beautiful design of cars and arenas, interface and the menu, and also gameplay.}\\

    \item \textbf{Tristan Deloche:}{
    The (gay) Deserter}\\

    \item \textbf{Cyril Olivier:}{
    Cyril is 20 years old, originating from Paris. He studied in France then information technology and programming at the swiss Federal Polytechnic School of Lausanne. He is the normal lion, intelligent and strong. With programming his passion, he readies himself to face his greatest challenge: dethroning the alpha lion...hum, ok later. Until then, bidding his time he roars towards all and prepares to help to the best of his abilities, to incredible journey ahead.}\\
    
\end{itemize}


\chapter{In-depth explanation of concept and gameplay}


\chapter{Technological overview}


\chapter{Economical overview}


\chapter{Timetables}


\section{First Submission}


\section{Second Submission}


\section{Third Submission}


\chapter{Conclusion}

\chapter{Bibliography}
